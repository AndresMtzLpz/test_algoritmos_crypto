\documentclass{IEEEtran}

\usepackage[left=1.5cm,right=1.5cm]{geometry}
\usepackage[
spanish,
es-nodecimaldot,
es-tabla
%english
]{babel}
\usepackage[utf8]{inputenc}
\usepackage[T1]{fontenc}
\usepackage{float}
\usepackage{crossreftools}
\usepackage{graphicx}
\usepackage{grffile}
\usepackage{longtable}
\usepackage{wrapfig}
\usepackage{rotating}
\usepackage[normalem]{ulem}
\usepackage{amsmath}
\usepackage{textcomp}
\usepackage{amssymb}
\usepackage{capt-of}
\usepackage{hyperref}
\usepackage{minted}
\usepackage{subfiles}
\usepackage{caption}
\usepackage{subcaption}
\usepackage[acronym, toc]{glossaries}

\usepackage{fancyhdr}
\usepackage{graphicx}
\usepackage[table,xcdraw]{xcolor}
\usepackage{multicol}
\usepackage{tabularx,booktabs}
\usepackage{siunitx}

\usepackage{grffile}
\usepackage{longtable}
\usepackage{wrapfig}
\usepackage{rotating}
\usepackage[normalem]{ulem}
\usepackage{amsmath}
\usepackage{textcomp}
\usepackage{amssymb}
\usepackage{capt-of}
\usepackage{booktabs}
\usepackage{hyperref}
\usepackage{caption}

\definecolor{LightGray}{gray}{0.9}
\definecolor{DarkGray}{HTML}{191919}
\definecolor{custom}{HTML}{F8F8F8}

\newenvironment{code}{\captionsetup{type=listing}}{}

\usemintedstyle{emacs}

\usepackage[ruled,vlined]{algorithm2e}


\renewcommand{\listingscaption}{Código}
\renewcommand\listoflistingscaption{Índice de \listingscaption\@s}

\setminted[python]{frame=single,framesep=1mm,baselinestretch=0.5,breaklines=true,bgcolor=custom,fontsize=\scriptsize}
\setminted[shell-session]{frame=single,framesep=1mm,baselinestretch=0.5,breaklines=true,bgcolor=custom,fontsize=\scriptsize}
\graphicspath{
  % {./img_common}
  {./img}
}
% \usepackage[T1]{fontenc}
% \renewcommand*\familydefault{\sfdefault} %% Only if the base font of the document is to be sans serif

% \pagestyle{fancy}
% \fancyfoot[R]{\thepage}
% \fancyfoot[C]{\includegraphics[width=0.05\textwidth]{inge_logo}}
% \fancyhead[L]{\leftmark}
% \fancyhead[R]{\rightmark}


\usepackage{authblk}

\begin{document}
\title{Un título    }
\author{
  \IEEEauthorblockN{
    Romero Andrade Cristian\IEEEauthorrefmark{1},
    Romero Andrade Vicente\IEEEauthorrefmark{2},
    Author Three\IEEEauthorrefmark{3} and
    Author Four\IEEEauthorrefmark{4}
  }
  \IEEEauthorblockA{Department of Whatever,
    Facultad de Ingeniería\\
    Email: \IEEEauthorrefmark{1}\href{mailto:cristian.romero@comunidad.unam.mx}{cristian.romero@comunidad.unam.mx},
    \IEEEauthorrefmark{2}\href{mailto:dark_reggae_93@comunidad.unam.mx}{dark\_reggae\_93@comunidad.unam.mx},
    \IEEEauthorrefmark{3}author.three@add.on.net,
    \IEEEauthorrefmark{4}author.four@add.on.net}}

\maketitle{}

\tableofcontents{}

\hypertarget{test-de-algoritmos-criptograficos}{%
  \section{Test de algoritmos
    criptograficos}\label{test-de-algoritmos-criptograficos}}

\hypertarget{instrucciones-de-uso}{%
  \subsection{Instrucciones de uso}\label{instrucciones-de-uso}}

Hay que tener docker y docker-compose instalado

Se corre el siguiente comando la primera ves que se utiliza

\begin{verbatim}
$ docker-compose build
\end{verbatim}

Se corre el siguente comando cada que se quiera ejecutar

\begin{verbatim}
$ docker-compose up
\end{verbatim}

Los resultados de la ejecucion se encontrarán en el directorio
\href{./results}{results}

\hypertarget{descripciuxf3n}{%
  \subsection{Descripción}\label{descripciuxf3n}}

Es un proyecto para la materia de criptografia semestere 2021-2

\hypertarget{contenido}{%
  \subsubsection{Contenido}\label{contenido}}

\hypertarget{algoritmos-a-probar}{%
  \paragraph{Algoritmos a probar}\label{algoritmos-a-probar}}

\begin{itemize}
        \tightlist
  \item
        AES-EBC 256 bits
  \item
        AES-CBC 256 bits
  \item
        SHA-2 384 bits
  \item
        SHA-2 512 bits
  \item
        SHA-3 512 bits
  \item
        RSA-OAEP 1024 bits
  \item
        RSA-PSS 1024 bits
  \item
        DSA 1024 bits
  \item
        ECDSA Prime Field 521 bits
  \item
        ECDSA Binary Field 571 bits
\end{itemize}

\hypertarget{operaciones-a-probar}{%
  \paragraph{Operaciones a probar}\label{operaciones-a-probar}}

\begin{itemize}
        \tightlist
  \item
        Cifrado
  \item
        Decifrado
  \item
        Hashing
  \item
        Firmado
  \item
        Verificación
\end{itemize}

\hypertarget{clasificaciuxf2n-de-operaciones}{%
  \subsubsection{Clasificaciòn de
    operaciones}\label{clasificaciuxf2n-de-operaciones}}

\hypertarget{cifrado-y-decifrado}{%
  \paragraph{Cifrado y decifrado}\label{cifrado-y-decifrado}}

En esta clasificaciòn se encuentran:

\begin{itemize}
        \tightlist
  \item
        AES-EBC 256 bits
  \item
        AES-CBC 256 bits
  \item
        RSA-OAEP 1024 bits
\end{itemize}

Este ultimo no es comparable con AES-EBC y CBC

\hypertarget{hashing}{%
  \paragraph{Hashing}\label{hashing}}

En esta clasificaciòn se encuentra:

\begin{itemize}
        \tightlist
  \item
        SHA-2 384 bits
  \item
        SHA-2 512 bits
  \item
        SHA-3 512 bits
\end{itemize}

\hypertarget{firma-y-verificacion}{%
  \paragraph{Firma y verificacion}\label{firma-y-verificacion}}

Aqui se encuentran:

\begin{itemize}
        \tightlist
  \item
        RSA-PSS 1024 bits
  \item
        DSA 1024 bits
  \item
        ECDSA Prime Field 521 bits
  \item
        ECDSA Binary Field 571 bits
\end{itemize}


\end{document}
