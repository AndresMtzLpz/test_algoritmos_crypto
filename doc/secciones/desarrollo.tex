\documentclass[../main.tex]{subfiles}
\begin{document}

\hypertarget{test-de-algoritmos-criptograficos}{%
  \subsection{Test de algoritmos
    criptográficos}\label{test-de-algoritmos-criptograficos}}

\hypertarget{instrucciones-de-uso}{%
  \subsubsection{Instrucciones de uso}\label{instrucciones-de-uso}}

Hay que tener docker y docker-compose instalado

Se corre el siguiente comando la primera ves que se utiliza


\begin{code}
\begin{minted}{shell-session}
$ docker-compose build
\end{minted}
\end{code}

Se corre el siguiente comando cada que se quiera ejecutar

\begin{code}
\begin{minted}{shell-session}
$ docker-compose up
\end{minted}
\end{code}

Los resultados de la ejecución se encontrarán en el directorio
\href{./results}{results}

\hypertarget{descripciuxf3n}{%
  \subsubsection{Descripción}\label{descripciuxf3n}}

Es un proyecto para la materia de criptografía semestre 2021--2

\hypertarget{algoritmos-a-probar}{%
  \paragraph{Algoritmos a probar}\label{algoritmos-a-probar}}

\begin{itemize}
        \tightlist{}
  \item
        AES-EBC 256 bits
  \item
        AES-CBC 256 bits
  \item
        SHA-2 384 bits
  \item
        SHA-2 512 bits
  \item
        SHA-3 512 bits
  \item
        RSA-OAEP 1024 bits
  \item
        RSA-PSS 1024 bits
  \item
        DSA 1024 bits
  \item
        ACEDAS Prime Field 521 bits
  \item
        ECDSA Binary Field 571 bits
\end{itemize}

\hypertarget{operaciones-a-probar}{%
  \paragraph{Operaciones a probar}\label{operaciones-a-probar}}

\begin{itemize}
        \tightlist{}
  \item
        Cifrado
  \item
        Descifrado
  \item
        Hashing
  \item
        Firmado
  \item
        Verificación
\end{itemize}

\hypertarget{clasificaciuxf2n-de-operaciones}{%
  \paragraph{Clasificación de
    operaciones}\label{clasificaciuxf2n-de-operaciones}}

\hypertarget{cifrado-y-decifrado}{%
  \paragraph{Cifrado y descifrado}\label{cifrado-y-decifrado}}

En esta clasificación se encuentran:

\begin{itemize}
        \tightlist{}
  \item
        AES-EBC 256 bits
  \item
        AES-CBC 256 bits
  \item
        RSA-OAEP 1024 bits
\end{itemize}

Este ultimo no es comparable con AES-EBC y CBC

\hypertarget{hashing}{%
  \paragraph{Hashing}\label{hashing}}

En esta clasificación se encuentra:

\begin{itemize}
        \tightlist{}
  \item
        SHA-2 384 bits
  \item
        SHA-2 512 bits
  \item
        SHA-3 512 bits
\end{itemize}

\hypertarget{firma-y-verificacion}{%
  \paragraph{Firma y verificación}\label{firma-y-verificacion}}

Aquí se encuentran:

\begin{itemize}
        \tightlist{}
  \item
        RSA-PSS 1024 bits
  \item
        DSA 1024 bits
  \item
        ECDSA Prime Field 521 bits
  \item
        ECDSA Binary Field 571 bits
\end{itemize}


\end{document}
